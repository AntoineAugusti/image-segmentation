% L'option handout permet de supprimer la barre de navigation
\documentclass[handout]{beamer}
\usepackage[utf8]{inputenc}
\usepackage[french]{babel}
\usepackage[T1]{fontenc}
\usepackage{amsmath}
% Pour pouvoir insérer des images
\usepackage{graphicx}
\usepackage{wrapfig}
\graphicspath{images/}
% Gestion des couleurs
\usepackage{color}
\definecolor{red}{RGB}{231, 76, 60}

% Un joli thème flat
\usetheme{Rochester}

% Personnalisation du thème
\usecolortheme[named=red]{structure}
% Numéro de slides dans le footer
\setbeamertemplate{footline}[frame number]
\setbeamertemplate{blocks}[shadow=false]

% ------------------------------------ %
% -- METADONNÉES DU DOCUMENT --------- %
\title{
	A quad-tree approach to image segmentation which combines statistical and spatial information
}
\author{
	Manon \textsc{Ansart} \\
	\vspace{5px}
	Antoine \textsc{Augusti}
}
\date{7 janvier 2015}

% Générer une page de titre à chaque début de section
\AtBeginSection[]
{
	\begin{frame}[plain]
	\frametitle{Sommaire}
	\tableofcontents[currentsection, hideothersubsections]
	\end{frame}
}

% Début du document
\begin{document}

	% Génération de la page de titre
	\begin{frame}[plain]
		\titlepage
	\end{frame}

	% Génération du sommaire
	\begin{frame}[plain]
		\frametitle{Sommaire}
		\tableofcontents
	\end{frame}


	% //////////////////////////////// %
	% /// La segmentation //////////// %
	\section{La segmentation}

		%% Principe général
		\subsection{Principe général}
		\begin{frame}
			\frametitle{Principe général}

			La segmentation est une opération qui a pour but de rassembler des pixels entre eux suivant des critères pré-définis.\\
			\vspace{10px}
			\textbf{Segmentation statistique}
			\begin{itemize}
				\item Par seuillage : choix de plusieurs seuils pour assigner une classe à chaque pixel. Maximisation de l'entropie, méthode d'Otsu\dots
				\item Par classification : algorithme des K-moyennes par exemple.
			\end{itemize}

			\vspace{10px}
			\textbf{Segmentation spatiale}
			\begin{itemize}
				\item Approche par croissance de région ou par décomposition / fusion.
			\end{itemize}

		\end{frame}
\end{document}