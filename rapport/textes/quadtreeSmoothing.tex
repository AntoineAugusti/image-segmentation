La première étape de l'algorithme de segmentation d'image est un lissage effectué à l'aide d'un quadtree. Nous allons d'abord voir comment un quadtree peut être utilisé dans ce contexte puis comment le lissage est effectué.

\subsection{Utilisation d'un quadtree}
	Afin de former notre quadtree, nous allons commencer par créer le niveau $k = 0$ de l'arbre, c'est-à-dire les feuilles. Ce niveau va contenir tous les pixels de l'image, chaque valeur étant stockée dans un des nœuds. Puis on forme les niveaux supérieurs en faisant la moyenne des nœuds par groupe de 4. Chaque nœud n'étant pas une feuille a donc pour la valeur la moyenne de ses quatre fils. Chaque niveau de l'arbre donne donc une nouvelle image lissée.\\

	Prenons une image de taille $N \times N$, avec $N = 2^m$. On note $d(i,j)$ le pixel de coordonnées $(i,j)$ de l'image, avec  $0 \leq i \leq N$ et $0 \leq j \leq N$. On note $q(i,j,k)$ le nœud de l'arbre correspondant au pixel de coordonnée $(i,j)$ de l'image au niveau $k$ de l'arbre, avec $0 \leq i \leq 2^{m-k}$, $0 \leq j \leq 2^{m-k}$ et $0 \leq k \leq m$.\\

	Le rang $k = 0$ correspond simplement aux données, on a donc \[q(i,j,0) = d(i,j)\]

	Puis chaque nœud a pour valeur la moyenne de ses quatre fils, donc pour $k > 0$,
	\[ q(i,j,k) = \frac{q(2i, 2j, k-1) + q(2i+1, 2j, k-1) + q(2i, 2j+1, k-1) + q(2i+1, 2j+1, k-1)}{4}\]

	% petit schéma montrant quel nœud de l'arbre correspond à quel pixel et la réduction de la taille de l'image

	On remarque que plus k augmente et plus l'image est lisse (un pixel a pour valeur la moyenne de 4 autres) et est de petite taille. Il nous suffit dont de prendre un niveau suffisant de l'arbre pour avoir une image lisse. La question reste de déterminer le niveau optimal $m'$ pour lequel le lissage est suffisant sans pour autant que l'image soit trop dégradée pour effectuer les étapes suivantes de segmentation. Dans la partie suivant nous verrons comment déterminer ce niveau $m'$.

\subsection{Détermination du niveau optimal}
	Afin de trouver le niveau de lissage maximum $m'$ pour lequel on peut toujours effectuer la segmentation, on pose $r = 2^n$, le rayon minimal des régions que l'on souhaite détecter et qui doit être fixé a priori.

	En coupant le quadtree au niveau $m'$, on pourra détecter toutes les régions de rayon supérieur à $2^{m'+1}$. Pour détecter les régions de rayon $r=2^n$, on veut donc
	\[ 2^{m'+1} \leq 2^n \]
	\[ \Leftrightarrow m'+1 \leq n \]
	\[ \Leftrightarrow m' \leq n-1 \]

	On cherche à lisser notre image le plus possible tout en gardant une résolution suffisante pour détecter les régions de rayon $r$, on veut donc maximiser $m'$ tout en respectant $m' \leq n-1$, on a donc $m' = n-1$.