Afin d'étudier dans de bonnes conditions notre article scientifique, il est nécessaire de définir certains termes qui seront utilisés dans la suite du présent rapport.

\subsection{La segmentation d'image}
	\enquote{La segmentation d'image est une opération de traitement d'images qui a pour but de rassembler des pixels entre eux suivant des critères pré-définis. Les pixels sont ainsi regroupés en régions, qui constituent un pavage ou une partition de l'image.}\cite{wikiSegmentationImage} On peut par exemple vouloir distinguer des objets du fond de l'image. Le terme de \enquote{binarisation} est utilisé quand on sépare une image en deux classes.\\

	La segmentation d'image est un des thèmes les plus courants en traitement d'images aujourd'hui car la mise au point d'algorithmes de segmentation de haut niveau reste un véritable challenge. À ce jour, les principales méthodes de segmentation sont au nombre de quatre :
	\vspace{10px}
	\begin{enumerate}
		\item Segmentation fondée sur les régions (\textit{region-based segmentation}). On y retrouve deux méthodes : la croissance de région (\textit{region-growing}) et la décomposition / fusion (\textit{split and merge}).
		\item Segmentation fondée sur les contours (\textit{edge-based segmentation}) ;
		\item Segmentation par classification (\textit{classification}) ou par le seuillage des pixels en fonction de leur intensité (\textit{thresholding}) ;
		\item Segmentation fondée par une utilisation commune des trois premières segmentations.
	\end{enumerate}

\subsection{La segmentation spatiale d'image}
	La segmentation spatiale d'image comprend les méthodes de segmentation fondée sur les régions (\textit{region-based segmentation}) et de la segmentation fondée sur les contours (\textit{edge-based segmentation}). Les méthodes de segmentation fondées sur les régions reposent sur l'homogénéité de certaines propriétés comme l'intensité (le niveau de gris d'un pixel), la texture et la position. D'un autre côté, les méthodes de segmentation fondées sur les contours utilisent principalement le gradient pour trouver les contours des objets. Certains méthodes utilisent une combinaison de ces deux méthodes, comme la croissance de région et la décomposition / fusion.

	\subsubsection{La croissance de région}
		L'algorithme de croissance de région commence par un pixel, choisi arbitrairement. Les itérations de l'algorithme s'intéressent ensuite aux pixels voisins du pixel courant et examinent les similarités avec les pixels déjà ajoutés à la région courante. Quand l'ajout d'un pixel n'est plus possible, car ne respectant pas les critères définis, une nouvelle région est créée et continuera de croître ensuite.
	
	\subsubsection{La décomposition / fusion}
		L'approche de décomposition / fusion est l'opposée de la croissance de région. Initialement, on crée une unique région englobant toute l'image. L'homogénéité est alors évaluée ; si le critère d'homogénéité n'est pas satisfait, la région est alors décomposée. Cette étape est répétée plusieurs fois, de telle manière que l'image soit découpée en une multitude de régions. L'autre étape importante de cet algorithme est l'étape de fusion. On cherche des couples de régions candidates à une fusion et on les note en fonction de l'impact que cette fusion aurait sur l'image. On fusionne alors les couples de régions les mieux notés, et on réitère jusqu’à ce que les caractéristiques de l'image remplissent une condition prédéfinie : nombre de régions, luminosité, contraste ou texture générale donnée, ou alors jusqu’à ce que les meilleures notes attribuées aux couples de régions n'atteignent plus un certain seuil.