Afin d'étudier dans de bonnes conditions notre article scientifique, il est nécessaire de définir certains termes qui seront utilisés dans la suite du présent rapport.

\subsection{La segmentation d'image}
	\enquote{La segmentation d'image est une opération de traitement d'images qui a pour but de rassembler des pixels entre eux suivant des critères pré-définis. Les pixels sont ainsi regroupés en régions, qui constituent un pavage ou une partition de l'image.}\cite{wikiSegmentationImage} On peut par exemple vouloir distinguer des objets du fond de l'image. Le terme de \enquote{binarisation} est utilisé quand on sépare une image en deux classes.\\

	La segmentation d'image est un des thèmes les plus courants en traitement d'images aujourd'hui car la mise au point d'algorithmes de segmentation de haut niveau reste un véritable challenge. À ce jour, les principales méthodes de segmentation sont au nombre de quatre :
	\vspace{10px}
	\begin{enumerate}
		\item Segmentation fondée sur les régions (\textit{region-based segmentation}). On y retrouve deux méthodes : la croissance de région (\textit{region-growing}) et la décomposition / fusion (\textit{split and merge}).
		\item Segmentation fondée sur les contours (\textit{edge-based segmentation}) ;
		\item Segmentation par classification (\textit{classification}) ou par le seuillage des pixels en fonction de leur intensité (\textit{thresholding}) ;
		\item Segmentation fondée par une utilisation commune des trois premières segmentations.
	\end{enumerate}