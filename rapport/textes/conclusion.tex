Suite à l'analyse de cet article scientifique, nous sommes maintenant conscient des différentes techniques de segmentation qui existent et de leurs spécificités générales. En étudiant l'algorithme décrit dans cet article, nous avons vu comment l'utilisation des statistiques et des propriétés spatiales d'une image pouvaient permettre de définir plusieurs régions dans une image. Nous avons apprécié l'utilisation d'une structure de données originale, à savoir le quadtree, qui est l'élément sur lequel toutes les analyses sont effectuées.\\

Nous avons mis un certain temps à comprendre en détail toutes les parties de l'algorithme, en particulier les parties portant sur les estimations des erreurs, mais nous avons apprécié l'effort que nous avons effectué. Nous pensons pouvoir affirmer que nous connaissons maintenant les techniques de segmentation et que nous avons bien compris les différentes étapes de l'algorithme décrit dans l'article : lissage, classification statistique, estimation des frontières. Nous espérons avoir réussi à expliquer ces multiples étapes clairement dans ce rapport.